\documentclass[12pt]{article}
\usepackage{times}
\usepackage{latexsym}
\usepackage{graphicx}
\usepackage{url}
\usepackage{float}
\usepackage[table,xcdraw]{xcolor}

\linespread{1}
\title{Anthropocentric Bias in Viral Genome Sequencing: Which Viruses Get Sequenced?}
\date{\today}
\author{Jacob Osborne}


\begin{document}

    \begin{titlepage}
        \begin{center}
            \vspace*{1in}
            \LARGE
            Anthropocentric Bias in Viral Genome Sequencing: Which Viruses Get Sequenced?

            \vspace*{1in}
            \large
            Jacob Osborne

            \vfill
            [DEGREE PROGRAM] \\
            Dr. Claus Wilke, Integrative Biology \\
            \today
        \end{center}
    \end{titlepage}
    
    \begin{abstract}
        The NCBI viral genomes database provides an extensive catalogue of genetic
        information from a variety of virus species. Yet, not all viruses are equal
        in their relevance to human life; there is some suspicion that those more
        directly important to our lives are significantly more likely to be
        sequenced than those that are not. To ascertain the existence and extent
        of this anthropocentric bias, a large-scale statistical analysis of the
        database was performed, and showed this was indeed the case.
        Specific patterns as to which viruses are more likely  to be sequenced
        are also documented and discussed.
    \end{abstract}

    \section{Introduction}

    The NCBI Viral Genomes database provides a large catalogue of viral genomes 
    sequenced by scientists around the world. It is the preeminent resource for
    obtaining records of viral genomes for scientific analysis.

    Yet, there is some doubt as to the extent to which the genomes catalogued
    therein can be taken as representative of viral genomes present in the
    natural world as a whole. We suspect the genomes of viruses that infect
    humans, domestic animals, and domestic plants - living things that are
    directly relevant to human life - are far more likely to be sequenced than
    the genomes of those which do not.

    The aim of this project, therefore, is to ascertain the extent of this
    anthropocentric bias in the NCBI database.

    



\end{document}